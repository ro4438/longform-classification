% !TEX root = recommending-interesting-writing.tex
\begin{abstract}
  We describe a system for recommending nonfiction writing to editors at The Browser, a curation service for interesting writing. The editors' goal is to filter new articles from many RSS feeds to choose interesting nonfiction writing to share with readers. To aid the editors, we build a recommender system that classifies articles based on their content. The recommendation model is \gls{rfs}, chosen for its scalability and explainability, with architectures that allow editors to understand which words in an article informed a recommendation~\citep{altosaar2020rankfromsets:}. Further, editors can choose which latent features of articles to upweight. We show that \gls{rfs} performs well in classifying articles when evaluated on historical data, and conduct an online evaluation with qualitative feedback from the editors to show that the system performs well in practice on unseen articles. Due to resource constraints, we deploy \gls{rfs} using a microservices architecture on a cloud computing platform. For reproducibility we open source the end-to-end pipeline and release a demo\footnote{\url{https://the-browser.github.io/recommending-interesting-writing/}}, training and deployment scripts, and model parameters.\footnote{\url{https://github.com/the-browser/recommending-interesting-writing}}
\end{abstract}