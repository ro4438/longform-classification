% !TEX root = set_recommendation.tex
\begin{abstract}
  Recommendation models suggest items to users. We study a variant of this
  problem where each item has an unordered collection of attributes, such as
  tags on an image, user upvotes of a post, or foods in a meal. In each case, an
  item has a handful of attributes, but the number of possible items and
  attributes can be in the thousands or millions. This large number of items and
  attributes makes recommending items with attributes difficult, and requires a
  scalable recommendation model to share information across items with similar
  sets of attributes.
  %
  We solve the problem of recommending items with attributes starting with a
  simple observation. We connect the performance of a recommendation model to
  the objective function it is trained with. One way to assess the performance
  of a recommendation model is with recall, which is binary: a model does or
  does not recall an item. This means that fitting a recommendation model's
  parameters using classification can maximize the evaluation metric of recall.
  We develop \acrlong{rfs}, a deep classification model for recommending items
  with attributes. This model treats item attributes as side information and
  uses embeddings to discriminate items a user has consumed from items a user is
  unlikely to consume. We scale \acrlong{rfs} to large datasets using a
  stochastic optimization-based negative sampling procedure, and show that it
  outperforms competing approaches. Furthermore, the learned embeddings reveal
  interpretable structure in user behavior on both datasets. We show that
  \acrlong{rfs} is a general approach, by applying a theorem developed in
  \citet{zaheer2017deep} to prove that \acrlong{rfs} can approximate other
  recommendation models that operate on items with attributes, including
  principled approaches such as content-based matrix factorization.
  %
  These results demonstrate that if a practitioner has data where items have
  attributes and measures recommendation performance with recall, building a
  classifier such as \acrlong{rfs} is the best course of action.
  
  % Code for is available at \url{https://github.com/altosaar/rankfromsets}.
  % Matrix factorization approaches that handle side-information exist, but they
  % cannot handle large attribute vocabularies. We consider recommending items
  % to users in the setting where items have bags of attributes.

  % % problem setup; why is this interesting; motivation - this data occurs often
  % Every day, users consume items made up of sets of attributes.
  % % examples of such data
  % A user might consume a meal that is a set of foods, or a document containing a
  % set of words.
  % % the task
  % We focus on building recommendation models that produce rankings of items
  % using their sets of attributes.
  % % there are many possible sets of attributes
  % But recommending items described by sets is difficult because of the large
  % number of sets.
  % % foreshadow desiderata
  % This means we cannot posit a model with unique parameters for
  % every set, and requires a model to share information across items with similar
  % sets of attributes.
  % % implicitly, sets are order invariant
  % Models whose output does not depend on the order that set elements are fed to
  % the model form the class of order-invariant models.
  % % contribution: we show that existing models the reader is used to fall in
  % % this class
  % We show that this class of order-invariant models encompasses existing
  % recommendation models such as matrix factorization and
  % permutation-marginalized recurrent neural networks.
  % % one way of evaluating is recall
  % One way to assess the performance of a recommendation model is with recall.
  % % contribution: recall is binary, we note that this leads to classification
  % Recall is inherently binary: a model does or does not recall an item.
  % % our algorithm is to fit the parameters via classification
  % This means we can fit a recommendation model's parameters using classification
  % to maximize the evaluation metric of recall. 
  % % contribution: we show it approximates other algorithms
  % We build \acrshort{rankfromsets}, a recommendation model that can approximate
  % any order-invariant model such as matrix factorization, and that theoretically
  % maximizes recall.
  % % contribution: we build a classifier with negative sampling
  % The model is a classifier parameterized by a neural network,
  % \acrshort{rankfromsets}, and we develop an algorithm that trains the model to
  % distinguish items a user consumes from items a user is unlikely to consume.
  % % contribution: we show it does well
  % Empirically, we show that this model outperforms other models on recommending
  % meals from the Lose It! app and recommending documents from the arXiv.
  % % Interpretability
  % We also find that the model reveals interpretable patterns in this user
  % behavior.
  % % code
  % Code for \acrshort{rankfromsets} is available at
  % \url{https://github.com/altosaar/rankfromsets}.
  % We consider the task of recommending items to users, where each item has a
  % set of attributes and user feedback is implicit.
  %   % Our goal is recommendation with implicit data, where each item has a set
  %   % of attributes.
  % For example, the item is a meal, the set a bag of foods composing the meal,
  % and the user implicitly prefers the meal by choosing to eat it. We develop
  % \acrshort{rankfromsets} (RFS), a flexible recommendation model that
  % transforms the recommendation problem into classification. RFS learns
  % attribute embeddings and an order-invariant prediction function that is used
  % to rank candidate items. Training an RFS model is performed through negative
  % sampling in a manner that provably maximizes recall.
  %   % Since our data is implicit, the negative labels need to be defined; we
  %   % draw from the uniform distribution over items. This algorithm and
  %   % negative sampling distribution is theoretically justified.
  %   % We prove that such a classifier maximizes recall.
  % \acrshort{rankfromsets} outperforms competing methods in empirical
  % evaluation and learn interpretable embeddings. Code is available at
  % \href{https://github.com/altosaar/rankfromsets}{https://github.com/altosaar/rankfromsets}.
\end{abstract}

%%% Local Variables:
%%% mode: latex
%%% TeX-master: "set_recommendation"
%%% TeX-engine: xetex
%%% End: