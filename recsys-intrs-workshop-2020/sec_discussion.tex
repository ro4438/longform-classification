% !TEX root = set_recommendation.tex
\section{Discussion}

The task of recommending items with attributes is difficult for several reasons.
It is unclear how to incorporate set-valued side information into models that
scale to large numbers of items and attributes. In addition, existing
recommendation models that leverage item attributes are not directly tied to
evaluation metrics. We developed \acrlong{rfs}, a scalable recommendation model
for items with attributes. Theoretically, we showed that optimizing its loss
function optimizes recall, and that it can approximate a class of recommendation
models including content-based matrix factorization. Empirically, \gls{rfs}
outperformed competitors and scaled to large datasets.

\paragraph{Simple theory and implementation} It is surprising that \gls{rfs}
outperforms the collaborative topic Poisson factorization model in
\citet{gopalan2014content-based}, as \gls{rfs} is a much simpler model.
\gls{rfs} does not require approximate posterior inference as in \gls{ctpf}, but
mere binary classification. \gls{rfs} is also simpler in terms of
implementation: a scalable example is a few dozen lines of python in
\Cref{sec:code}, compared to several thousand lines of C++ released by the
authors of \gls{ctpf}.

\paragraph{Architectures} There is a wealth of deep learning models that can be
used to parameterize \gls{rfs}. For example, attention-based transformer
networks can operate on set-valued input~\cite{lee2018set-transformer} and are
an interesting choice of architecture for \gls{rfs}.
% larger batch sizes on cpu
% For good performance on the arXiv dataset, large batch sizes were necessary. We
% were limited in batch size due to our use of a GPU. Reimplementing the model in
% C++ would free the model from GPU memory restrictions, enabling larger batch
% sizes and better performance.
% optimization: ctpf benefits from coordinate ascent
% \paragraph{Optimization} Collaborative topic Poisson factorization as the latter
% enjoys an efficient optimization algorithm. The conjugacy properties of the
% matrix factorization model permit coordinate ascent and fast convergence in few
% iterations. The nonconvex objective function that \gls{rfs} relies on may be
% amenable to certain optimization techniques, such as dual coordinate
% ascent~\citep{yu2011dual}, and we leave this to future work.

\paragraph{Side information} The \gls{rfs} model can easily accommodate various
metadata about users, items, or attributes. Attribute-level or user-level side
information is particularly interesting, and when available, should lead to
improved recommendation performance.

% we did not study regularization
\paragraph{Regularization} Further performance gains from regularizing \gls{rfs}
are also interesting avenues for future work. L2 regularization for sparsity and
dropout for decorrelation may yield improvements.

% RMSProp~\cite{graves2013generating} was also tested when fitting
% \gls{rfs} to the arXiv data. This optimizer outperformed
% collaborative topic Poisson factorization in terms of both in-matrix precision
% and recall, but matched the performance of the matrix factorization model in
% terms of out-matrix precision and recall. This means there are tradeoffs of
% using different optimizers that depend on the data and evaluation metrics of
% interest.
% initializing to word2vec
\paragraph{Pretrained embeddings} We note that the word embedding baseline
performed slightly worse as collaborative topic Poisson factorization, which is
surprising given that the word embedding model is much simpler. Further
performance gains in \gls{rfs} should be possible by initializing user and item
attribute embeddings to pretrained word embeddings as in \citet{chen2017joint}.

% gap in prop 2
\paragraph{Theory (approximation)} \Cref{prop:universal-approximation} means
that \gls{rfs} can approximate the output of any order-invariant model
such as matrix factorization. However, there is no guarantee that fit with
maximum likelihood estimation using the objective in \Cref{eq:objective},
\gls{rfs} \emph{will} approximate a specific model. We leave the development of
such approximation techniques to future work.
% out-of-vocabulary words
% An open question is how to account for new attributes, for example in
% crowdsourced data. An extension of this work would be to combine it with
% character-level embeddings as in \cite{bojanowski2017enriching} to enable the
% model to better perform recommendations for out-of-vocabulary attributes.
% can we optimize metrics other than recall?
\paragraph{Theory (ranking metrics)} How well does binary classification perform
for other ranking-based recommendation metrics, such as non-discounted
cumulative gain? Analyzing this question is more difficult, and we leave this to
future work. We conjecture that a different loss function should allow a similar
proof to \Cref{prop:maximizing-recall}.

\paragraph{Theory of generalization} With sufficient data, \acrlong{rfs} can
learn arbitrary distributions of users consuming items with attributes. But
performance on finite data can vary. Developing generalization theory for
\acrlong{rfs} remains an open question.

% \section*{Acknowledgments}
% % and emotional support
% J.A. is grateful to Mark Goldstein and Bharat Srikisan for helpful
% discussions, and to Kyle Cranmer
% and the Center for Data Science at New York University for desk space. The
% experiments presented in this article were performed on computational resources
% supported by the Princeton Institute for Computational Science and Engineering
% and the Office of Information Technology's High Performance Computing Center and
% Visualization Laboratory at Princeton University.

%%% Local Variables:
%%% mode: latex
%%% TeX-master: "set_recommendation"
%%% End: