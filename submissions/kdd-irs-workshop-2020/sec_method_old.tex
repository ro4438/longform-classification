\acrshort{rankfromsets} is a classifier whose output is used to rank items for a
user $u$ based on each item's set of attributes $x_m \in I$ and learned user
embeddings $\theta_u$. Every item $x_m \in I$ is represented by a subset of
attributes $v \in V$ from a vocabulary of attributes $V$. Item $m$ is
represented by the set of attributes $x_m \subseteq V$.

The \acrshort{rankfromsets} model is a classifier parameterized by a neural
network $f$ that predicts whether user $u$ consumed item $m$, demarcated by the
binary indicator $\yum = 1$. The model is trained to discriminate items a user
consumed, $(x_m, \yum=1)$, from items a user is unlikely to consume,
$(x_k, \yuk=0)$. In implicit feedback data, only the former datapoints
$(x_m, \yum = 1)$ are observed. (We describe how to define negative samples
$(x_k, \yuk=0)$ later.)

The classifier outputs the probability of a user $u$ consuming an item $m$:
\begin{align}
p(\yum = 1 \mid x_m) = \sigma\left(f\left(\theta_u, \sum_{v\in x_m}
  \beta_v,~~h(x_m)\right)\right).
\label{eq:rankfromsets}
\end{align}
The learned user embedding $\theta_u$, item attribute embeddings $\beta_v$, and
item intercept function $h(x_m)$ are used to make the prediction of item
consumption. The item intercept function maps items to
intercepts based on the item attributes (we describe how to construct this
function later). The sigmoid function is denoted $\sigma$ and squashes the
output of the neural network to be between $0$ and $1$, yielding the probability
that user $u$ consumes item $m$.

Fitting the model amounts to learning the user embeddings, the item attribute
embeddings, parameters of the item intercept function, and any parameters of the
neural network $f$ such as weights and biases. We define the loss function in
\Cref{sec:fitting}.

Does \acrshort{rankfromsets} satisfy the modeling criteria laid out in
\Cref{sec:desiderata}? The model is order-invariant: the summation of the item
attribute embeddings $\beta_v$ and the construction of the item intercept
function (detailed below) guarantees that the model output is invariant to the
order of set elements. Second, the model provably improves recall, as described
in \Cref{sec:theory}. Third, the choice of neural network $f$ to parameterize
the model enables universal approximation of other order-invariant models.
Finally, the model satisfies parameter-sharing, as the item attribute embeddings
$\beta_v$ and the intercept function $h$ are shared across items and allow the
model to scale.

\paragraph{Theoretical results.} \acrshort{rankfromsets} is supported by
propositions in \Cref{sec:theory}. We derive a result related to the
recommendation performance of the model. A common evaluation metric for
recommendation models is recall. Recall can be viewed as a binary task, and we
prove that this immediately leads to the conclusion that a classifier such as
\acrshort{rankfromsets} can maximize this metric. The second result concerns
universal approximation. The structure of the neural network used to
parameterize \acrshort{rankfromsets} enables it to approximate other models such
as matrix factorization or permutation-marginalized recurrent neural networks.

\subsection{Cold start and collaborative filtering}

The cold start problem arises when new items or new users appear.
Recommendations of such items or for such users are only possible if there
exists metadata about items (such as sets of attributes) or metadata about users
(such as age). As \acrshort{rankfromsets} recommends items based on their
attributes, we focus on the cold start setting of new items, where there is item
metadata available.

\paragraph{\acrshort{rankfromsets} item attribute embeddings leaven the cold
  start problem.} If an item has not been consumed by any user, it is considered
a `cold start` item, as the recommendation model does not have any historical
information with which to rank the item. The model must use the item attributes
to inform the rankings of such cold start items. \acrshort{rankfromsets}
addresses the cold-start problem by leveraging item attribute embeddings, which
allow the ranking of new items through their sets of attributes. Consider the
recommendation of cold start items that contain common attributes, such as meals
with commmon foods (e.g. bananas, tomatoes, bread), or arXiv papers that use
standard terms (e.g. recommendation, deep, learning). Item attribute embeddings
for such attributes that are shared across items enable information to be shared
between items with zero consumption data and items that have been consumed and
have similar attributes.

As our focus is on the setting where users have a history of previously-consumed
items, we need to ensure that \acrshort{rankfromsets} supports collaborative
filtering. In other words, the model needs to make recommendations for a user by
filtering items based on the consumption patterns of other users. 

\paragraph{Collaborative filtering via item intercepts.}
\acrshort{rankfromsets} shares information across users through the item
intercept function $h$. The item intercept function enables the ranking of items
for a user based on similar users' patterns of consumption. The item intercept
function is shared across users, allowing one user's consumption choices to
inform another's ranking. This is in contrast to the item attribute embeddings,
which depend on the attributes of every item (these content-based attributes
cannot capture patterns in usage independent of content). The parameterization
of the item intercept function $h$ depends on the size of the data. If the data
is small enough, $h$ can function as a lookup for unique intercepts for every
item. However, if the number of items is so large that unique item intercepts
lead to overfitting, we can define $h$ using additional information about every
item. For example, if the data consists of foods in meals, we can define a meal
intercept as the sum of food intercepts, yielding a scalable item intercept
function. These two choices of parameterization satisfy the desideratum of
order-invariance. The input to the item intercept function is a set of
attributes, so the above constructions (using item-level parameters or sums)
ensure every item intercept is independent of the order of the item's set of
attributes.

\subsection{Neural network architectures}
\label{sec:parameterizations}
There are several choices for parameterizing the neural network $f$ in the
\acrshort{rankfromsets} recommendation model. The following choices of neural
network architectures enable universal approximation of other order-invariant
models; we outline the technical requirements for this in \Cref{sec:theory}. In
these parameterizations, the neural network $f$ outputs logits that predict item
consumption; these are used to train the classifier. For recommendation, items
are ranked using the logits.

In the inner product parameterization, the classifier makes predictions using
the dot product of the user embedding $\theta_u$ and the sum of the item
attribute embedding sand item intercept function.  This architecture is equivalent to
a neural network with user-dependent weights $\theta_u$. The neural network
outputs logits that predict whether user $u$ consumed item $m$:
\begin{align}
  f\left(\theta_u, x_m\right) = \theta_u^\top\left(\sum_{v\in x_{m}}
  \beta_v + h(x_m)\right)
  \label{eq:inner-product}
\end{align}

As an alternative parameterization, $f$ can be a deep neural network:
\begin{align}
  f(\theta_u, x_m) = \phi\left(\theta_u, \sum_{v\in x_m}
  \beta_v, h(x_m)\right),
  \label{eq:neural-network}
\end{align}
where the deep network $\phi$ has weights and biases and takes as inputs the
user embedding, sum of item attribute embeddings, and item intercept. This
architecture contrasts the inner product parameterization: ex ante, it is
unclear whether a finite-depth, finite-size neural network can learn the inner
product function.

A third parameterization is a combination of the above, using an idea borrowed
from residual networks~\citep{DBLP:journals/corr/HeZRS15}. A neural network can
be used to learn the residual in the inner product model:
\begin{align}
  f\left(\theta_u, x_m\right) = \theta_u^\top\left(\sum_{v\in x_m}
  \beta_v + h(x_m)\right) + \phi(\theta_u, x_m).
  \label{eq:residual}
\end{align}

We discuss the generalization and universal approximation capabilities of these
parameterizations in the following section.

\subsection{Fitting the model}
\label{sec:fitting}
\acrshort{rankfromsets} is fit to data using a classification objective function
and a stochastic optimization algorithm.

The model can be represented using a Bernoulli distribution, with generative
process $\yum~\sim~\textrm{Bernoulli}\left(\yum; \sigma(f(u, x_m))\right)$. The
probability of a user consuming an item is given by \Cref{eq:rankfromsets}.

As \acrshort{rankfromsets} is a classifier, it requires both positively- and
negatively-labeled training data. But in implicit feedback data, there are no
negative labels. Users explicitly consume items $(x_m, \yum = 1)$, but do not
indicate dislike of items $(x_m, \yuk = 0)$. For training the
\acrshort{rankfromsets} classifier, we assign negative labels to items a user is
unlikely to consume. These items are called negative samples. Negative samples
are drawn uniformly at random from the collection of items $I$. (We justify this
choice of negative sampling distribution in the next section.)

Denote the parameters of the model by $\theta$, and let $K$ be the number of
negative samples. The model parameters are the user embeddings, item attribute
embeddings, parameters of the item intercept function, and any weights and
biases of the neural network in \Cref{eq:rankfromsets}. The maximum likelihood
objective function for a single datapoint $(x_m, \yum=1)$ is
\begin{align}
\begin{split}
  \cL(\theta, & (x_m, \yum), \{(x_k, \yuk)\}) =\\
  &\log p(\yum = 1 \mid x_m; \theta) + \sum_{k=1}^K \log p(\yuk = 0 \mid x_k;
  \theta),\\
  &\textrm{with}~x_k\sim\textrm{Uniform}(I).
  \label{eq:objective}
\end{split}
\end{align}

% \begin{align}
%   \cL_{\textrm{total}}\left(\theta, \{(x_m,
%         \yum)\}\right) = & \sum_u \sum_{m: \yum = 1}\cL(\theta, (x_m, \yum)) 
% \end{align}
%\begin{algorithm}[t]
  \DontPrintSemicolon \KwIn{Initial parameters $\theta_0$, number of negative
    samples $K$, learning rate $\rho$}

  \KwOut{Parameters $\theta$ of the \acrshort{rankfromsets} model that maximize 
    recommendation performance}

  \While{Validation recall improves} {
    Sample a datapoint $(x_m, \yum=1)$ uniformly at random \\
    \For{k from 1 to K}{ Draw negative sample $x_k \sim \textrm{Uniform}(I)$;
      assign $\yuk = 0$ }
    $\theta_{t+1} = \theta_t + \rho\nabla  \cL(\theta, (x_m, \yum), \{(x_k, \yuk)\})$\\

  } \Return{$\theta$}\;
\caption{\textbf{The \acrshort{rankfromsets} algorithm.} The per-datapoint objective
    $\cL$ is defined in \Cref{eq:objective}. }
\label{algo:rankfromsets}
\end{algorithm}


%%% Local Variables:
%%% mode: latex
%%% TeX-master: "set_recommendation"
%%% End:


\paragraph{Optimization.} We use stochastic gradient ascent~\citep{Robbins:1951}
to maximize the objective. This algorithm subsamples datapoints
$(x_m, \yum = 1)$, and optimizes the objective with respect to the model
parameters by taking stochastic gradient steps in the parameter space. We use
minibatches of datapoints to improve convergence.

\section{theory}
\label{sec:theory}

Having exhibited several parameterizations of the \acrshort{rankfromsets}
recommendation model, we prove that these parameterizations satisfy the
desideratum of improved recall in \Cref{sec:desiderata}. By noting that recall
is a binary task, it follows that a classifier such as \acrshort{rankfromsets}
can maximize this metric (\Cref{prop:1}). We also prove that
\acrshort{rankfromsets} can approximate any order-invariant recommendation model
such as matrix factorization (\Cref{prop:2}).

Recall measures how accurately a recommendation model ranks items consumed by a
user. It is computed as follows. First, for a specific user $u$, use the model
to rank all items $x_m \in I$. Denote the top $M$ items returned by the model as
${x_1, x_2, \ldots, x_M}$. Recall is defined as the fraction of items the user
$u$ consumed in these top-ranked items:
\begin{align}
  \textrm{Recall@}M &= \frac{\lvert \{x_m : \yum = 1 \} \rvert }{M}
\label{eq:recall}                      
\end{align}

\subsection{Classifiers maximize recall}
\label{sec:optimal}

The central insight here is that recall can be viewed as classification: a
recommendation model does or does not recall an item consumed by a user. This
connection between the performance of a recommender and its parameterization as
a classifier forms the scaffolding for \acrshort{rankfromsets}.

Datapoints in a classifier are of the form $(x, y) \sim F$ where $F$ is the
empirical data-generating distribution. The generalization error of a binary
classifier $p(y \mid x)$ is the misclassification rate,
\begin{equation}
E = \sum_{x \in F} \left[ \hat y(x) \neq y \right] \, .
\end{equation}
\begin{proposition}
  A binary classifier is trained with positive examples from a user's history
  and negative samples drawn uniformly from the collection of items $I$. If such
  a classifier achieves zero generalization error, then it achieves maximum
  recall.
  \label{prop:1}
\end{proposition}
\begin{proof}
  A model with zero generalization error is a perfect classifier; it assigns
  greater probability to positively-labeled datapoints than to
  negatively-labeled datapoints. In other words, it ranks positive examples
  above negative examples. Recall is measured by the fraction of
  positively-labeled items in a ranking returned by the model. In a classifier
  that achieves zero generalization error, positively-labeled datapoints muts be
  ranked higher than other datapoints, maximizing recall.
  %Recall is maximized up to negative samples being
%  identical to positive examples (such examples will have the same rank).
\end{proof}

\Cref{prop:1} connects recommender systems to classification. This connection may
seem vacuous or tautological. But the message is subtle---in terms of recall, a
good classifier makes for a good recommender.

This theory also justifies the choice of negative samples for implicit feedback
data in the objective, \Cref{eq:objective}. For good recall performance, it is
sufficient to assign negative labels uniformly over the items $I$. Note that the
distribution of negative samples can be other than uniform, as long as every
item in the collection is assigned positive weight. In the infinite data limit,
any such distribution of negative samples will result in a classifier with
maximum recall.

\subsection{\acrshort{rankfromsets} can approximate any order-invariant model}

Having shown that a classifier such as \acrshort{rankfromsets} can yield good
performance as a recommendation model in terms of recall, we turn to the model's
flexibility. The desideratum of order-invariance for models that rank from sets
in \Cref{sec:desiderata} leads to a specific class of models. We show that
\acrshort{rankfromsets} can approximate any model in this class.

The criterion of order-invariance in \Cref{sec:desiderata} requires that a
recommendation model that takes as input a set of attributes be invariant to the
order the set elements are fed to the model. We formally define this class of
recommendation models as order-invariant models, and show that
\acrshort{rankfromsets} can approximate any model in this class.

\paragraph{Order-invariant models.} A set is unordered, and therefore invariant
to permutation. Recommendation models that rank items from their sets of
attributes must be invariant to the order that set elements are fed to the
model. For a model $f$ that is used to rank item $m$ for user $u$, order
invariance with respect to the attributes $v_i \in x_m$ is written
\begin{align}
  f(u, {v_1, v_2, v_L}) &= f(u, {v_{\pi(1)}, v_{\pi(2)}, \ldots
                                     v_{\pi(L)}})~\forall~\pi,
\label{eq:order-invariance}                                     
\end{align}
where $\pi$ is a permutation of the indices of individual attributes.

We can verify that a recommendation model is in the class of order-invariant
models by checking that it satisfies \Cref{eq:order-invariance}. Examples of
order-invariant models are matrix factorization, recommendation models based on
word embeddings, and permutation-marginalized recurrent neural networks.
We show that a Bayesian matrix factorization model \cite{Gopalana} is
order-invariant in \Cref{sec:order-invariance}. These models are evaluated in
\Cref{sec:experiments}.

The class of order-invariant models is large. The next proposition allows us to
focus on comparing architectures for a single model rather than comparing
models. This simplifies design of recommendation models evaluated using recall.
\acrshort{rankfromsets} is parameterized by neural networks, and \Cref{prop:2}
says that the model can approximate any other model in the class of
order-invariant models. This result is derived from the framing of the
parameterizations of \acrshort{rankfromsets} in \Cref{sec:parameterizations} as
neural networks with learned item-dependent covariates and user-dependent
weights, and use of the fact that neural networks are universal approximators.

\begin{proposition}
  Assume the vocabulary of attributes (set elements) is countable,
  $\lvert V \rvert < \lvert \bbN_0 \rvert$. \acrshort{rankfromsets}, with
  parameters on the order of the size of the vocabulary can approximate any
  order-invariant recommendation model.
  \label{prop:2}
\end{proposition}
The proof follows directly from Theorem~2 in
\citet{DBLP:journals/corr/ZaheerKRPSS17} and we will not restate it here. (The
only change to the proof is the mapping from set elements to one-hot vectors,
$c \colon V \to \left\{0, 1\right\}^{\lvert V \rvert}$ to yield a unique
representation of every member of the powerset.)

\Cref{prop:2} supports using \acrshort{rankfromsets} in practice: not only does
the model improve recall (\Cref{prop:1}), but it can also approximate other
models, including matrix factorization. Matrix factorization and other
order-invariant models may not have theoretical guarantees of improved
evaluation metrics or universal approximation. Focusing on building efficient
architectures to parameterize \acrshort{rankfromsets} rather than designing
extensions of models may simplify the design of recommenders.

\subsection{Comparing parameterizations of \acrshort{rankfromsets} using
  generalization}
\label{sec:generalization}

Is there an optimal architecture for \acrshort{rankfromsets} between the inner
product, \Cref{eq:inner-product}, or residual parameterization in
\Cref{eq:residual}? It depends. In the regime of infinite data and infinite
parameters, the inner product and residual architectures are equivalent:
\Cref{prop:2} states that given infinite data, both are universal approximators.
In theory, they can approximate any order-invariant model such as matrix
factorization. But in the finite data, finite parameter regime, there are
tradeoffs that may favor one architecture over another. Depending on assumptions
about the data-generating distribution and parameter-sharing choices, one
architecture may be more efficient than another.

With finite data and finite paramaters, the optimal parameterization of
\acrshort{rankfromsets} is dependent on the data-generating distribution. Recall
that observations of user-item interactions are generated by
$ \yum~\sim~\textrm{Bernoulli}\left(\yum; \sigma(f(u, x_m)\right)$, where $f$ is
a function that outputs logits. We describe how different assumptions about the
logit function $f$ lead to the residual architecture or inner product
architecture yielding the best predictive performance. Note that the number of
parameters across architectures must be equal for a fair comparison.

To illustrate a scenario where the inner product architecture outperforms the
residual architecture, suppose the logit function of the data-generating
distribution has the following form. With two-dimensional embeddings (and a
single attribute per item) say the logit function is given by
$f(u, x_m) = \theta_1\beta_1 + \theta_2\beta_2^{10}$. The second dimension of
the item attribute embedding $\beta$ has been raised to the power of $10$.
Consider fitting the inner product model with a dimensionality of $2$ to data
from this generative process. The optimal setting of the item attribute
embeddings is $\beta^* = (\beta_1, \beta_2^{10})$. Now consider fitting the
residual model. The requirement that the number of parameters be equal across
architectures forces the neural network to have less parameters. This limits the
ability of the residual architecture to learn that the second component of the
item attribute embedding should be $\beta_2^{10}$, as neural networks with small
hidden layer size may not be able to approximate complex polynomials. 

For an example where the residual model would outperform the inner product
model, consider the following generative process. The logit function is given by
$f(u, x_m) = \theta_u^\top \sum_{v\in \xum}\beta_v + \eta(\theta_u, \sum_{v\in
  \xum} \beta_v)$. Here $\eta$ is some nonlinear function. This generative
process mirrors \Cref{eq:residual} and is best approximated by the residual
model. We verified both of these claims with simulation studies.

% How can we ensure our choice of model does not lead to overfitting and enables
% prediction of held-out data? In recommendation data, one way to define
% generalization is to consider the matrix of observations of shape users by items
% $(U, I)$. As $U\rightarrow \infty$ and $I\rightarrow \infty$, it is clear that a
% model can memorize the training data if the number of parameters is
% $\mathcal{O}(UI)$. As long as the number of model parameters grows slower than
% $U$ or $I$, and the model predicts well, generalization is possible.

The above two examples show that the choice of architecture in
\acrshort{rankfromsets} is data-dependent. To ensure that the model does not
overfit as new users or items are included in the training data, we need to
compare the number of parameters to the number of datapoints. A model with
parameters the size of the training data can overfit by memorizing the training
data. For generalization to be possible, overfitting can be avoided if the
number of parameters grows slower than the size of the data. The technical
backing for this comes from asymptotic statistics and the concept of sieved
likelihoods. Specifically, the maximum likelihood estimation procedure with the
objective function in \Cref{eq:objective} can be replaced by maximization of a
sieved likelihood function. The `sieve' refers to filtering information as the
number of parameters (in this case, user and item representations) grows with
the number of observations. The sieved likelihood function enables the analysis
of asymptotic behavior as the number of users grows $U\rightarrow \infty$ and
the number of items grows $I\rightarrow \infty$, An example of a technique to
grow the number of parameters in a way that supports generalization is given in
Chapter 25 of \citet{vaart_1998}.

%%% Local Variables:
%%% mode: latex
%%% TeX-master: "set_recommendation"
%%% End: