% !TEX root = recommending-interesting-writing.tex
\begin{abstract}
  We describe a system for recommending nonfiction writing to editors at The Browser, a curation service for interesting writing. The editors' goal is select articles from a large list of candidates, and these selections of nonfiction writing are shared with subscribers. To aid the editors, we build a recommender system that classifies articles based on their content. The recommendation model is \gls{rfs}, chosen for its scalability and explainability, with architectures that allow editors to understand which words in an article informed a recommendation~\citep{altosaar2020rankfromsets:}. Further, editors can choose which latent features of articles to upweight. We train \gls{rfs} on historical data, and show that this translates to good performance in an online setting with qualitative feedback from editors on unseen candidate articles. Due to resource constraints, we deploy \gls{rfs} using a microservices architecture on a cloud computing platform. For reproducibility and transparency of the user-facing system, we open source the end-to-end pipeline and release a demo\footnote{\url{https://the-browser.github.io/recommending-interesting-writing/}}, data collection, training and deployment scripts, and model parameters.\footnote{\url{https://github.com/the-browser/recommending-interesting-writing}}
\end{abstract}