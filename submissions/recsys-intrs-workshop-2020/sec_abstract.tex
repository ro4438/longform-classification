% !TEX root = recommending-interesting-writing.tex
\begin{abstract}
  We build a visual interface for recommending articles to editors at The Browser, a curation service for interesting writing. From a large list of candidates, editors decide which articles are selected and shared with subscribers. To aid the editors in this decision-making task, we build a visual interface for a recommendation model, \gls{rfs}~\citep{altosaar2020rankfromsets:}, that classifies articles based on their words. Control of the recommendation model is built into the visual interface. For example, an editor can use a topic slider to receive a new list of recommendations according to topical words in articles. These topic sliders might be used to increase or decrease the ranking of articles with words related to crime, business, or technology. The visual interface is also designed to be explanation-aware: words that contribute positively or negatively to an article's ranking are displayed. For the backend of the visual interface, \gls{rfs} is trained on historical data. In an offline empirical study, we find that \gls{rfs} outperforms \acrshort{bert}~\citep{devlin2019bert:}, a competitive classification model, in terms of recall. Further, we measure \gls{rfs} to be 10 times faster to train and to return predictions 2000 times faster than \acrshort{bert}. These are beneficial properties for the visual interface, and we demonstrate that \gls{rfs} can be deployed on the free tier of AWS Lambda using a short python script and numpy dependency. For reproducibility, transparency, and trust of the visual interface, we open source and release a public demonstration,\footnote{\url{https://the-browser.github.io/recommending-interesting-writing/}} data collection, training and deployment scripts, and model parameters.\footnote{\url{https://github.com/the-browser/recommending-interesting-writing}}
\end{abstract}